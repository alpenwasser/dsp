\documentclass{minimal}
\usepackage[a4paper,margin=5mm,landscape]{geometry}
\setlength{\parindent}{0pt}
\usepackage{pgfplots}
\pgfplotsset{compat=1.14}
%http://tex.stackexchange.com/questions/13816/dimension-too-large-while-plotting-with-pgfplots#13838
\input{params.tex}
\begin{document}
\begin{tikzpicture}
    \pgfplotsset{every axis/.style={
        height=0.5\textheight,
        width=0.5\textwidth,
        unbounded coords=discard,
        filter discard warning=false,
        xmin=\plotStart,
        xmax=\plotStop,
    }}
    % Analog Input Signal
    \begin{axis}[at={(0mm,0.5\textheight)},each nth point=\eachNth]
        \addplot[red] table {analogSignal.dat};
    \end{axis}
    % Sampled Signal
    \begin{axis}[at={(0.5\textwidth,0.5\textheight)}]
        \addplot+[ycomb,draw=gray,mark options={gray}] table {sampledSignal.dat};
        \addplot[red,no markers] coordinates {(\plotStart,0) (\plotStop,0)};
    \end{axis}
    % Error between reconstructed signal and original signal
    \begin{axis}[at={(0mm,0mm)}, each nth point=\eachNth, restrict y to domain=\yLowerBound:\yUpperBound]
        \addplot[red] table {errorSignal.dat};
    \end{axis}
    % Reconstructed signal overlaid with original signal
    \begin{axis}[at={(0.5\textwidth,0mm)}, each nth point=\eachNth]
        \addplot[red] table {analogSignal.dat};
        \addplot[blue] table {reconstructedSignal.dat};
    \end{axis}
\end{tikzpicture}
\end{document}
